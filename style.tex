\usepackage[T1]{fontenc}
\usepackage[utf8]{inputenc}
\usepackage{lmodern} % times
\usepackage[english]{babel}

\usepackage{ragged2e} %to justify text

\usepackage{xparse}
\usepackage{amsmath,amssymb,wasysym}
\usepackage{pgfpages}
\usepackage[]{xcolor}
\usepackage{framed}

\usepackage{tikz}
\usetikzlibrary{shapes,automata,trees,backgrounds,arrows,fit,positioning}
%\tikzstyle{initial}+=[initial text=]
\tikzset{onslide/.code args={<#1>#2}{%
  \only<#1>{\pgfkeysalso{#2}} % \pgfkeysalso doesn't change the path
}}

  \tikzset{
    invisible/.style={opacity=0},
    visible on/.style={alt={#1{}{invisible}}},
    alt/.code args={<#1>#2#3}{%
      \alt<#1>{\pgfkeysalso{#2}}{\pgfkeysalso{#3}} % \pgfkeysalso doesn't change the path
    },
  }



% --------- STYLE -------------------

\newenvironment{variableblock}[3]{%
  \setbeamercolor{block body}{#2}
  \setbeamercolor{block title}{#3}
  \begin{block}{#1}}{\end{block}}

\usetheme{default} %Madrid,Copenhagen,Warsaw,CambridgeUS,Boadilla
\usecolortheme{default} %dolphin,beaver,rose,orchid

% \usefonttheme{professionalfonts} %structurebold
% \useoutertheme{wuerzburg}
% \useinnertheme{rectangles}

% \renewcommand\mathfamilydefault{\rmdefault} %anstellt von mathserif
\let\Tiny=\tiny


% ---------- Colors --------
\definecolor{darkgray}{rgb}{0.33, 0.33, 0.33}

\definecolor{myred}{RGB}{255,15,0} % my red
\definecolor{myorange}{RGB}{255,147,0} % my orange
\definecolor{myblue}{rgb}{0.23,0.4,0.7} % my blue

\definecolor{frametitlecolor}{named}{darkgray}
\definecolor{maincolor}{named}{OrangeRed} %OrangeRed
\definecolor{accentcolor}{named}{OrangeRed} %RedOrange
\definecolor{accentcolor2}{named}{myorange} % never used

\definecolor{thmcolor}{named}{OrangeRed}
\definecolor{examplecolor}{named}{myorange}


% ---------- Beamer Fonts --------
\usefonttheme{serif}
\setbeamerfont{frametitle}{shape=\mdseries,family=\sffamily}
\setbeamerfont{framesubtitle}{shape=\mdseries,family=\sffamily}
\setbeamerfont{frametitle continuation}{shape=\mdseries,family=\sffamily}
\setbeamerfont{footline}{family=\sffamily}
\setbeamerfont{footnote mark}{family=\sffamily}
\setbeamerfont{section in toc}{family=\sffamily}
\setbeamerfont{alerted text}{shape=\bfseries,family=\sffamily}
\setbeamerfont{title}{family=\sffamily}
\setbeamerfont{enumerate item}{shape=\bfseries,family=\sffamily}
\setbeamerfont{enumerate subitem}{shape=\bfseries,family=\sffamily}
\setbeamerfont{page number in head}{shape=\bfseries,family=\sffamily}
\setbeamerfont{page number in foot}{shape=\bfseries,family=\sffamily}
%\setbeamerfont{}{series=\rmseries}


% ---------- Beamer Colors --------
\setbeamercolor{frametitle}{fg=frametitlecolor} 
\setbeamercolor{title}{fg=maincolor}
\setbeamercolor{item}{parent=title}
\setbeamercolor{section in toc}{parent=title}
\setbeamercolor{footline}{fg=frametitlecolor}
\setbeamercolor{alerted text}{fg=maincolor}

% \setbeamertemplate{blocks}[rounded][shadow=false]

%  \setbeamertemplate{frametitle}{\vspace*{-1pt}{\insertframetitle\par} 
%   \ifx\insertframesubtitle\@empty\else
%   \vspace*{-1pt}%
%   \insertframesubtitle\par\fi
%   \vspace*{-13pt}
%     \hspace*{0pt}\rule{\textwidth}{0.25pt}
%     \vskip-5pt
%     }
% \setbeamersize{text margin left=2ex}

% ---------- Beamer Footer --------
\setbeamertemplate{footline}
{
\hbox{\begin{beamercolorbox}[wd=1.0\paperwidth,ht=2.25ex,dp=1ex]{frametitlecolor}%title in head/foot
\usebeamerfont{footerFont}{
%\vspace*{-0.5ex}
\hspace*{2ex}
{\hspace*{10ex} \insertshorttitle\ -- \insertshortauthor \hfill \insertshortdate\ \hfill \insertframenumber\ of \inserttotalframenumber}}%/\inserttotalframenumber
\hspace*{2ex}
\end{beamercolorbox}%
}%
\vskip0pt%
}%

% ---------- Beamer Navbar --------
\setbeamertemplate{navigation symbols}{}

% ---------- Beamer TOC --------
\AtBeginSection[]
{
 \begin{frame}<beamer>
 \frametitle{Outline}
 \tableofcontents[currentsection,currentsubsection,hideothersubsections,subsectionstyle=show/shaded]
\end{frame}
}

\AtBeginSubsection[]
{
 \begin{frame}<beamer>
 \frametitle{Outline}
 \tableofcontents[currentsection,currentsubsection,hideothersubsections,subsectionstyle=show/shaded]
\end{frame}
}

% ---------- Highlights and Alerts --------
\newcommand{\alertsf}[1]{{\sffamily\color{maincolor}#1}}
\newcommand{\alertrm}[1]{{\rmfamily\color{maincolor}#1}}
\newcommand{\alertaccentboldsf}[1]{{\bfseries\sffamily\color{accentcolor}#1}}
\newcommand{\alertaccentboldrm}[1]{{\bfseries\rmfamily\color{accentcolor}#1}}
\newcommand{\alertaccentsf}[1]{{\sffamily\color{accentcolor}#1}}
\newcommand{\alertaccentrm}[1]{{\rmfamily\color{accentcolor}#1}}
\newcommand{\boldsf}[1]{{\bfseries\sffamily#1}}
\newcommand{\boldrm}[1]{{\bfseries\rmfamily#1}}

\newcommand{\descitem}[1]{\item[\bfseries\sffamily#1]}
\newcommand{\descitemaccent}[1]{\item[\bfseries\sffamily\color{accentcolor}#1]}
\newcommand{\itemaccent}{
  \usebeamercolor{itemize item}%
  \colorlet{fgColor}{fg}%
  \colorlet{bgColor}{bg}%

  \usebeamercolor{itemize subitem}%
  \colorlet{fgColor2}{fg}%
  \colorlet{bgColor2}{bg}%

\setbeamercolor{itemize item}{fg=accentcolor}
\setbeamercolor{itemize subitem}{fg=accentcolor}
\item
\setbeamercolor{itemize item}{fg=fgColor}
\setbeamercolor{itemize subitem}{fg=fgColor2}
}

\newcommand{\enumaccent}{
  \usebeamercolor{enumerate item}%
  \colorlet{fgColor}{fg}%
  \colorlet{bgColor}{bg}%

  \usebeamercolor{enumerate subitem}%
  \colorlet{fgColor2}{fg}%
  \colorlet{bgColor2}{bg}%

\setbeamercolor{enumerate item}{fg=accentcolor}
\setbeamercolor{enumerate subitem}{fg=accentcolor}
\item
\setbeamercolor{enumerate item}{fg=fgColor}
\setbeamercolor{enumerate subitem}{fg=fgColor}
}

\renewcommand{\emph}[1]{\alertaccentboldsf{#1}}
\newcommand{\accentc}[1]{{\color{accentcolor}#1}}



% boxes
\newenvironment{shadebox}
{
  \colorlet{shadecolor}{black!10}
  \begin{shaded}
}
{\end{shaded}}

\newenvironment{accentbox}
{
  \colorlet{shadecolor}{accentcolor!20}
  \begin{shaded}
}
{\end{shaded}}

% example box

\renewenvironment{example}
{
  \colorlet{shadecolor}{examplecolor!20}
  \begin{shaded}
    {\bfseries\sffamily\color{examplecolor}Example.}
}
{\end{shaded}}

% \NewDocumentCommand{\ex}{m}{
%   \colorlet{shadecolor}{accentcolor!20}
%   \begin{shaded}
%     \alertaccentboldsf{Example.} #1  
%   \end{shaded}
% }


% theorem box

\RenewDocumentEnvironment{theorem}{o}
 {%
  \colorlet{shadecolor}{black!10}
  \begin{shaded}\justifying
     % <code>
    \IfNoValueTF{#1}
    {{\bfseries\sffamily\color{thmcolor}Theorem.}\ }
    {{\bfseries\sffamily\color{thmcolor}Theorem\ {\mdseries(#1)}.}\ }%
    % <code>
 }
 {
  \end{shaded}
 }

